\documentclass[a4paper,12pt]{jsarticle}

%%%%%%%%%%%%%%%% パッケージの指定 %%%%%%%%%%%%%%%%
\usepackage[top=20truemm,bottom=40truemm,left=40truemm,right=40truemm]{geometry}
\usepackage{amsmath,amssymb}
\usepackage{bm}
\usepackage{braket}
\usepackage{cases}
\usepackage{dsfont}
\usepackage[dvipdfmx]{hyperref,graphicx}
\usepackage{pxjahyper}
\usepackage{fancyhdr}
\usepackage{here}
\usepackage{listings}
%%%%%%%%%%%%%%%%%%%%%%%%%%%%%%%%%%%%%%%%%%%%%%%%

%%%%%%%%%%% 式番号を(節番号.連番)に設定 %%%%%%%%%%%
\makeatletter
\@addtoreset{equation}{section}
\def\theequation{\thesection.\arabic{equation}}
\makeatother
%%%%%%%%%%%%%%%%%%%%%%%%%%%%%%%%%%%%%%%%%%%%%%%%

%%%%%%%%%%%%%% サンプルコード表示設定 %%%%%%%%%%%%%%
\renewcommand{\lstlistingname}{ソースコード}
\lstset{
  basicstyle={\ttfamily},
  identifierstyle={\small},
  commentstyle={\smallitshape},
  keywordstyle={\small\bfseries},
  ndkeywordstyle={\small},
  stringstyle={\small\ttfamily},
  frame={tb},
  breaklines=true,
  columns=[l]{fullflexible},
  numbers=left,
  xrightmargin=0zw,
  xleftmargin=3zw,
  numberstyle={\scriptsize},
  stepnumber=1,
  numbersep=1zw,
  lineskip=-0.5ex
}
%%%%%%%%%%%%%%%%%%%%%%%%%%%%%%%%%%%%%%%%%%%%%%%%

%%%%%%%%%%%%%%%% 文書の表題を設定 %%%%%%%%%%%%%%%%
\title{計算物理学II 第4回レポート課題}
%\author{氏名\\学籍番号\\所属}
\date{提出期限 : 2026年1月9日}
%%%%%%%%%%%%%%%%%%%%%%%%%%%%%%%%%%%%%%%%%%%%%%%%

\begin{document}

\maketitle

以下の課題\ref{sec:problem_svd},\ref{sec:problem_pi_darts},\ref{sec:problem_datacompression},\ref{sec:problem_usage_AI}に取り組み,その結果を\LaTeX でレポートにまとめよ.
なお,以下の点に留意せよ.
\begin{itemize}
    \item レポートにはタイトルを付け,氏名,学籍番号,所属,レポート作成日を記載すること.
    \item レポート作成時に,この.pdfファイルのソースファイル(\texttt{lecture10/report4/main.tex})を活用しても構わない.
    \item 読みやすく,体裁の整ったレポート作成を心がけて欲しい.
    \item 作成したレポートの.pdfファイルと{\bf 課題\ref{sec:problem_svd},課題\ref{sec:problem_pi_darts},課題\ref{sec:problem_usage_AI}で使用したソースコード(.pyファイル)}をmanabaに提出すること.
\end{itemize}

\section{}
\label{sec:problem_svd}

以下の行列の特異値を全て求めよ.
\footnote{
実は,{\bf (1)}から{\bf (4)}の中には具体的な特異値分解をせずとも答えが分かる行列がある.ぜひ考えてみて下さい.
}
\\

\begin{flushleft}
\qquad\qquad{\bf (1)}
$\displaystyle
    \begin{bmatrix}
        1 & 1 & 1 & 1 \\
        2 & 3 & 2 & -3 \\
        0 & 1 & 0 & 1 \\
    \end{bmatrix}
$
\qquad
{\bf (2)}
$\displaystyle
    \begin{bmatrix}
        1 & 2 \\
        2 & 1 \\
    \end{bmatrix}
$
\qquad
{\bf (3)}
$\displaystyle
    \frac{1}{\sqrt{2}}
    \begin{bmatrix}
        1 & 0 & 1  \\
        0 & \sqrt{2} & 0  \\
        1 & 0 & -1  \\
    \end{bmatrix}
$
\newline\newline\\

\qquad\qquad{\bf (4)}
対角成分が$2$,残りの成分は全て$1$で与えられる$2025$次の正方行列
\end{flushleft}

\section{}
\label{sec:problem_pi_darts}
図\ref{fig:circle}のように,一辺の長さが1の正方形の内部に存在する四分円を考える.
この正方形に向かってランダムにダーツを投げる.投げたダーツは必ず正方形の内部のどこかに刺さるものとする.
ダーツを投げた回数を$N$とし,投げたダーツが四分円の内部に刺さった回数を$M(<N)$とする.
ダーツの刺さる場所がランダムであれば,$N$が十分に大きい場合,$N$と$M$の比は正方形と四分円の面積の比になることが予想される.
つまり,
\begin{align*}
    \frac{M}{N}\sim\frac{\pi}{4}
\end{align*}
となることが期待できる.
したがって,図\ref{fig:circle}を的としてランダムにダーツを投げるだけで,円周率$\pi$を
\begin{align}
\label{eq:pi}
    \pi\sim4\frac{M}{N}
\end{align}
と推定することができる.
式\eqref{eq:pi}をプログラムを書いて確かめてみよう.

\begin{figure}[H]
    \centering
    \includegraphics[scale=0.4]{fig/circle.pdf}
    \caption{正方形内の四分円}
    \label{fig:circle}
\end{figure}

\noindent {\bf (A)}
$N=10^{2}, 10^{3}, 10^{4}$の三つの場合について,$K$回だけ$\pi$の推定値を求め,そのヒストグラムを作成し,レポートに掲載せよ.
そして,三つのヒストグラムを比較することで,得られる$\pi$の推定値と$N$の関係について議論せよ.
なお,$K$は$O(10^{3})$から$O(10^{4})$程度とし,各自で自由に決定せよ.
選んだ$N_{\rm sample}$の値についてはレポートに明記すること.~\\

\noindent {\bf (B)}
$N=2^{k}~(k=4,5,\cdots,19,20)$における$\pi$の推定値を求め,その結果を図にまとめよ.
ただし,横軸を$N$,縦軸を$\pi$の推定値とし,エラーバーをつけること.
また,得られた図から分かることを簡潔に述べよ.



\section{}
\label{sec:problem_datacompression}

$m$行$n$列の行列$A$の低ランク近似は,特異値分解によって以下のように与えられる.
\begin{align}
\label{eq:svd}
    A_{ij}
    =
    \sum_{k=1}^{\min(m,n)}
    U_{ik}\sigma_{k}V^{\dag}_{kj}
    \simeq
    \sum_{k=1}^{\chi}
    U_{ik}\sigma_{k}V^{\dag}_{kj}
    .
\end{align}
ここで,$\sigma_{k}$は$k$番目の特異値であり,降順($\sigma_{1}\ge\sigma_{2}\ge\cdots\ge\sigma_{\min(m,n)}$)で並んでいるものとする.
特異値の打ち切り次元$\chi$について,$\chi=\min(m,n)$ととれば,式\eqref{eq:svd}の右辺は元の行列$A$を厳密に再現するが,特異値の減衰が十分に早ければ$\chi<\min(m,n)$でも元の行列$A$を十分な精度で復元することができる.
このことを利用して,モノクロ画像データを圧縮してみよう.
手続きは以下の通りである.

\begin{enumerate}
    \item 画像ファイル(ここでは.jpgファイルとする)をモノクロ化し,モノクロ化された画像を行列とみなす.
    \item この行列に対し,特異値分解による低ランク近似を実行する.
    \item 低ランク近似で得られた行列をモノクロ画像として再構成する.
\end{enumerate}

この手続きに従って,\texttt{lecture10/report4/img}の中にある画像を圧縮し,どの程度の$\chi$で元の画像がどのくらい復元されるかPythonを使って調べてみよう.
ただし,以下の設問では,次のソースコード\ref{lib}のように各種ライブラリが\texttt{import}されていることを前提としている.\\
\begin{lstlisting}[caption=本課題で必要なライブラリの\texttt{import},label=lib]
import numpy as np
import matplotlib.pyplot as plt
from scipy import linalg
from PIL import Image
\end{lstlisting}
~\\

\noindent {\bf (A)}
ソースコード\ref{mono}は\texttt{filename}という名前の画像ファイルをモノクロ化し,モノクロ化された画像を行列へと変換するための関数である.
この関数を使って\texttt{image0.jpg}を行列へ変換した上で特異値分解を実行し,得られた特異値$\sigma_{i}$について,横軸を$i$,縦軸を$\sigma_{i}$とする図を作成せよ.

\begin{lstlisting}[caption=画像ファイルのモノクロ化とその行列への変換,label=mono]
def matrix_grayscale(filename):
    image = Image.open(filename)
    grayscale_image = image.convert("L")  
    return np.array(grayscale_image)
\end{lstlisting}
~\\

\noindent{\bf (B)}
元々の\texttt{image0.jpg}に対応する行列および,$\chi=5,25,50,100$で圧縮された行列をモノクロ画像として再構成するコードを作成し,得られたモノクロ画像をレポートに掲載せよ.
\footnote{
\LaTeX で複数の画像を見栄えよく掲載するには\texttt{minipage}環境を使うと良い.興味のある人は\texttt{minipage}環境の使い方を調べてみよ.
}
なお,ソースコード\ref{image}は与えられた行列\texttt{matrix\_image}をモノクロ画像として再構成するための関数である(引数の\texttt{matrix\_image}はNumPy配列にも対応している).
必要に応じて自由に使用してもらって構わない.
\begin{lstlisting}[caption=行列をモノクロ画像として再構成,label=image]
def plot_single_image(matrix_image):
    plt.figure(figsize=(8, 6))
    plt.plot()
    plt.title("Name")
    plt.imshow(matrix_image, cmap="gray")
    plt.axis("off")
    plt.savefig("visualized_image.pdf")
\end{lstlisting}
~\\

\noindent{\bf (C)}
\texttt{lecture10/report4/img}の中にある\texttt{image0.jpg}以外の画像,
\footnote{
\texttt{lecture10/report4/img}の中に格納されている\texttt{image5.jpg}は一様乱数を使って生成されたものである.
}
あるいは自分で好きに用意した.jpgファイルに対して,上の設問{\bf (A)},{\bf (B)}を行ってみよ.
ただし,圧縮を行う際の$\chi$の値は適宜変更しても構わない.

\section{}
\label{sec:problem_usage_AI}
この課題では,プログラミングにおける生成AIの活用方法について考えてみたい.
生成AIは,デバッグ(プログラムに混入したバグを特定・修正する作業)やリファクタリング(出力結果を変えずに,コードの可読性や保守性を向上させる書き換え)において,有用な支援ツールとなり得る.

実際に,生成AIを使って自身が書いたコードのリファクタリングをしてみよう.
ここでは,OpenAIが開発した対話型生成AIサービスであるChatGPT(\href{https://chatgpt.com/}{\texttt{https://chatgpt.com/}})を使ってみよう.
URLからアクセスすると,図\ref{fig:chatGPT}のような画面になる.
テキスト入力ウィンドウから生成AIへの質問を入力し,Enterキーを押すことで生成AIからの回答が得られる.
テキスト入力ウィンドウに図\ref{fig:question}のような入力をし,それに続いて,自身が作成したコードを貼り付けることで,生成AIにリファクタリングをさせることができる.

\begin{figure}[H]
    \centering
    \includegraphics[scale=0.14]{fig/chatGPT.png}
    \caption{ChatGPTの画面}
    \label{fig:chatGPT}
\end{figure}

\begin{figure}[H]
    \centering
    \includegraphics[scale=0.14]{fig/question.png}
    \caption{ChatGPTにリファクタリングをさせる}
    \label{fig:question}
\end{figure}

図\ref{fig:question}のような入力に続いて,第2回レポート課題4で自身が作成したソースコードを貼り付け,そのコードのリファクタリングを生成AIによって行ってみよ.
レポートには以下の二点について簡潔にまとめてもらえれば良い.

\begin{itemize}
    \item 生成AIが出力したコードが実行できたか.また,そのコードの出力結果は妥当であったか.
    \item 自身の書いたコードと生成AIが書いたコードを比較し,どのような差異(改善または改悪)が見られたか.
\end{itemize}

なお,この課題のソースコードについては,生成AIによってリファクタリングされたソースコードのみを提出すれば良い.


\section*{アンケート}

この項目は成績とは無関係です.
\href{https://forms.gle/kSkN8F4sz85EcZ6w6}{こちら(クリック)}のGoogleフォームよりアンケートにご協力ください.

%\bibliographystyle{unsrt}
%\bibliography{bib/ref}

\end{document}
