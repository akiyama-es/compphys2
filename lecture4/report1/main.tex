\documentclass[a4paper,12pt]{jsarticle}

%%%%%%%%%%%%%%%% パッケージの指定 %%%%%%%%%%%%%%%%
\usepackage[top=20truemm,bottom=40truemm,left=40truemm,right=40truemm]{geometry}
\usepackage{amsmath,amssymb}
\usepackage{bm}
\usepackage{braket}
\usepackage{cases}
\usepackage{dsfont}
\usepackage[dvipdfmx]{hyperref,graphicx}
\usepackage{pxjahyper}
\usepackage{fancyhdr}
\usepackage{here}
%%%%%%%%%%%%%%%%%%%%%%%%%%%%%%%%%%%%%%%%%%%%%%%%

%%%%%%%%%%% 式番号を(節番号.連番)に設定 %%%%%%%%%%%
\makeatletter
\@addtoreset{equation}{section}
\def\theequation{\thesection.\arabic{equation}}
\makeatother
%%%%%%%%%%%%%%%%%%%%%%%%%%%%%%%%%%%%%%%%%%%%%%%%

%%%%%%%%%%%%%%%% 文書の表題を設定 %%%%%%%%%%%%%%%%
\title{計算物理学II 第1回レポート課題}
%\author{氏名\\学籍番号\\所属}
\date{提出期限 : 2025年11月7日}
%%%%%%%%%%%%%%%%%%%%%%%%%%%%%%%%%%%%%%%%%%%%%%%%

\begin{document}

\maketitle

以下の課題\ref{sec:problem_1},\ref{sec:problem_2},\ref{sec:problem_3},\ref{sec:problem_4}に取り組み,その結果を\LaTeX でレポートにまとめよ.
なお,以下の点に留意せよ.
\begin{itemize}
    \item レポートにはタイトルを付け,氏名,学籍番号,所属,レポート作成日を記載すること.
    \item レポート作成時に,この.pdfファイルのソースファイル(\texttt{lecture4/report1/main.tex})を活用しても構わない.
    \item 読みやすく,体裁の整ったレポート作成を心がけて欲しい.
    \item 作成したレポートの.pdfファイルをmanabaに提出すること.
\end{itemize}

\section{}
\label{sec:problem_1}

好きな方程式を一つ書け.
また,その式がどういう方程式なのか簡潔な説明を書け.
例えば,式の中の記号の意味,項の意味,どのような分野で使われているか,それを解くと何が分かるか,などを書いてもらえれば良い.

\section{}
\label{sec:problem_2}

第2回演習で配布したPythonスクリプト(\texttt{lecture2/src/sample\_plot\_functions.py})を使って, 
\begin{align}
\label{eq:problem_2}
    f(x)=
    \exp\left[\left(x\sin(\alpha x)\right)^{2}\right]
\end{align}
のグラフを描け.
ただし,グラフの作成にあたっては,以下の点を満たすようにすること.
\begin{itemize}
    \item \texttt{sample\_plot\_functions.py}で定義されている「関数」\texttt{make\_plot}では二つの関数を同時にプロットできる.そこで,式\eqref{eq:problem_2}の関数について,(i)~$\alpha=1$の場合,および(ii)~$1<\alpha\le10$の場合の二つをプロットせよ.(ii)の場合については,$1<\alpha\le10$の範囲内で具体的な$\alpha$の値を一つ各自で決めること.なお,選んだ$\alpha$の値は必ずレポート中に明記せよ.
    \item 式\eqref{eq:problem_2}の関数を\texttt{sample\_plot\_functions.py}でプロットするためには,\texttt{cys}と\texttt{dys}の中身を編集する必要がある.例えば,(i)の場合,式\eqref{eq:problem_2}の関数をスクリプト上で実装するには,「\texttt{np.exp(0.5*(x*np.sin(x))**2)}」と書けば良い.
    \item \texttt{make\_plot}の\texttt{plt.title}を編集し,グラフのタイトルに式\eqref{eq:problem_2}を書け.スクリプト上で\LaTeX 表記を使う場合,「\texttt{r'\$}ここに数式を書く\texttt{\$'}」の記法に留意せよ(配布したスクリプト中の\texttt{plt.xlabel}や\texttt{plt.ylabel}の行で既にこの記法が使われている).
    \item \texttt{make\_plot}の\texttt{plt.plot}の\texttt{label}を編集し,「$\alpha=1$」のように$\alpha$の値をグラフ上に明記せよ.
    \begin{itemize}
        \item (任意)配布したスクリプトでは,「\texttt{'r-'}」(赤い実線),「\texttt{'b--'}」(青い破線)を用いているが,各自で\texttt{matplotlib}で使える線の種類や色を調べ,見やすいように自由に変更して良い.
    \end{itemize}
    \item \texttt{make\_plot}の\texttt{plt.title},\texttt{plt.xlabel}, \texttt{plt.ylabel}において,\texttt{fontsize}はすべて「12」にせよ.
    \item \texttt{make\_plot}でコメントアウトされている\texttt{plt.xlim}を有効にし,\texttt{plt.xlim(0,10)}と変更せよ.これにより,作成されるプロットの$x$軸の範囲が「0」から「10」までに固定される.
    \item \texttt{make\_plot}の\texttt{plt.figure}から\texttt{plt.savefig}までの間で,「\texttt{plt.yscale('log')}」の一文を加え,$y$軸を対数スケールに変更せよ.なお,Pythonスクリプトではインデント(空白)も意味を持つため,「\texttt{plt.yscale('log')}」の一行を挿入する際には,行頭が他の「\texttt{plt}」から始まる文と揃うようにせよ.
\end{itemize}
なお,(ii)で選んだ$\alpha$の値によっては,グラフが滑らかに描かれない場合がある.
その場合,下記の修正を試みよ.
\begin{itemize}
    \item \texttt{sample\_plot\_functions.py}では,\texttt{cys}と\texttt{dys}の中で関数の値が計算されているが,これらの関数の値は\texttt{cxs},\texttt{dxs}で指定された$x$軸の値でのみ計算される.
    配布したスクリプトでは,$0.0,~0.1,~0.2, \cdots, 9.9$のように,$0.1$刻みで分割された100個の$x$軸の値が設定されている.
    もし,プロットしたい関数が激しく振動する場合には,$x$軸の刻み幅を十分に小さく取ることで,その$x$依存性を高解像度で可視化できるはずである.
    例えば,0.01刻みで分割された1000個の$x$軸の値に修正してみると良い.
\end{itemize}
また,\LaTeX でレポートにまとめる際には,以下の点を満たすようにすること.
\begin{itemize}
    \item $y$軸を対数スケールに変更したことでどのような恩恵があったかを記述せよ(例えば,対数スケールを使わない場合と比較してみよ).
    \item $x$軸の刻み幅を変更した場合,それによってどのような恩恵があったかを記述せよ.
    \item 「\LaTeX 入門」の6節を参照し,図には「課題2の関数のプロット」というキャプションをつけること.
    \item 「\LaTeX 入門」の7節を参照し,\texttt{\textbackslash label}と\texttt{\textbackslash ref}を使って,「図$*$(図の番号)は$\circ\circ\circ$を表している.」といった文を本文中に書くこと.
\end{itemize}



\section{}
\label{sec:problem_3}

第2回演習で,日本の総人口の推移を表す図を\texttt{gnuplot}で作成した.
その時に使った人口推移のデータファイルには,各都道府県の人口推移のデータも含まれている.
このデータファイル(\texttt{lecture2/data/population.dat})を使い,好きな都道府県を一つ選んで,その都道府県の人口推移を表す図を作成せよ.
図の作成にあたっては,以下の点を満たすようにすること.
\begin{itemize}
    \item 図の横軸を西暦とし,縦軸を人口とすること.
    \item 人口は,総数,男性,女性の三つを同じ図にプロットし,それぞれに異なるシンボルと色を用いること.
    \item 何の図か一目で分かるように,軸のラベルと凡例(key)を付けること.
\end{itemize}
また,\LaTeX でレポートにまとめる際には,以下の点を満たすようにすること.
\begin{itemize}
    \item 「\LaTeX 入門」の6節を参照し,図の挿入時にキャプションをつけること.
    \item 「\LaTeX 入門」の7節を参照し,\texttt{\textbackslash label}と\texttt{\textbackslash ref}を使って,「図$*$(図の番号)は$\circ\circ\circ$の人口の推移を表している.」といった文を本文中に書くこと.
    \item 「\LaTeX 入門」の9節を参照し,使用したデータファイルの出典を参考文献として記載すること.
    \texttt{\textbackslash cite}を使って,「図$*$(図の番号)は国勢調査[$*$(参考文献の番号)]のデータに基づく.」といった文を本文中に書けば良い.
    また,参考文献情報については,このレポートと共に配布された.bibファイルを使って良い(\texttt{lecture4/report1/bib/ref.bib}).
\end{itemize}


\section{}
\label{sec:problem_4}

次のような$d$次元の運動量空間上の関数を考えよう.
\begin{align}
\label{eq:correlator_f}
    f(\bm{k})=\frac{1}{|\bm{k}|^{2}+a^{2}}
\end{align}
ここで,$\bm{k}=(k_{1},k_{2},\cdots,k_{d})$は$d$次元のベクトルであり,
$|\bm{k}|^{2}=\sum_{i=1}^{d}k_{i}^{2}$である.
$a$は正の実数とする.
以下の手順に沿って,式\eqref{eq:correlator_f}のFourier変換,
\begin{align}
\label{eq:correlator_g}
    g(\bm{r})
    =
    \left(\prod_{i=1}^{d}\int^{\infty}_{-\infty}\mathrm{d}k_{i}\right)
    \mathrm{e}^{\mathrm{i}\bm{k}\cdot\bm{r}}
    f(\bm{k})
\end{align}
を計算しよう.
なお,$\bm{r}=(r_{1},r_{2},\cdots,r_{d})$は$d$次元の位置ベクトルで,
$\bm{k}\cdot\bm{r}=\sum_{i=1}^{d}k_{i}r_{i}$
である.\\

\noindent {\bf (A)}
以下の等式を示せ.
\begin{align}
\label{eq:exp}
    \frac{1}{q}
    =
    \int^{\infty}_{0}\mathrm{d}t~\mathrm{e}^{-tq}
\end{align}
\\

\noindent {\bf (B)}
式\eqref{eq:exp}を使うと,式\eqref{eq:correlator_g}は,
\begin{align}
\label{eq:correlator_g_exp}
    g(\bm{r})
    =
    \left(\prod_{i=1}^{d}\int^{\infty}_{-\infty}\mathrm{d}k_{i}\right)
    \int^{\infty}_{0}\mathrm{d}t
    ~\mathrm{e}^{-t\left(|\bm{k}|^{2}+a^{2}\right)+\mathrm{i}\bm{k}\cdot\bm{r}}
\end{align}
と書き換えることができる.
ここで,積分の順序を交換し,各$k_{i}$に関する積分を先に実行しよう.
式\eqref{eq:correlator_g_exp}の各$k_{i}$積分が
\begin{align}
\label{eq:int_k}
    \int^{\infty}_{-\infty}\mathrm{d}k_{i}
    ~\mathrm{e}^{-tk_{i}^{2}+\mathrm{i}k_{i}r_{i}}
    =
    \sqrt{\frac{\pi}{t}}
    \exp\left[
        -\frac{r_{i}^{2}}{4t}
    \right]
\end{align}
となることを示せ.
ただし,以下のGauss積分は既知として良い.
\begin{align}
    \int^{\infty}_{-\infty}\mathrm{d}k
    ~\mathrm{e}^{-\alpha k^{2}}
    =
    \sqrt{\frac{\pi}{\alpha}}
\end{align}
\\

\noindent {\bf (C)}
式\eqref{eq:int_k}を使うと,式\eqref{eq:correlator_g_exp}は,
\begin{align}
\label{eq:correlator_g_bessel}
    g(\bm{r})
    =
    \pi^{d/2}
    \int^{\infty}_{0}\mathrm{d}t
    ~t^{-d/2}
    \exp\left[
        -a^{2}t-\frac{|\bm{r}|^{2}}{4t}
    \right]
\end{align}
となる.
この積分は第2種変形Bessel関数$K_{\nu}(z)$を使って表すことができる.
その積分表示は,
\begin{align}
    K_{\nu}(z)
    =
    \frac{1}{2}\left(\frac{z}{2}\right)^{\nu}
    \int^{\infty}_{0}\mathrm{d}t
    ~t^{-\nu-1}
    \exp\left[
        -t-\frac{z^{2}}{4t}
    \right]
\end{align}
である.
このとき,式\eqref{eq:correlator_g_bessel}が,
\begin{align}
    g(\bm{r})
    =
    2\pi^{d/2}a^{d-2}\left(\frac{2}{a|\bm{r}|}\right)^{d/2-1}
    K_{d/2-1}(a|\bm{r}|)
\end{align}
となることを示せ.
\\
%\clearpage

\noindent {\bf (D)}
$|\bm{r}|$が十分大きい場合,$K_{d/2-1}(a|\bm{r}|)$は,
\begin{align}
    K_{d/2-1}(a|\bm{r}|)
    \sim
    \frac{\mathrm{e}^{-a|\bm{r}|}}{\sqrt{a|\bm{r}|}}
\end{align}
の漸近形で書ける.
$|\bm{r}|$が十分大きい場合,$g(\bm{r})$の$|\bm{r}|$依存性が以下で与えられることを示せ.
\begin{align}
\label{eq:ornstein_zernike}
    g(\bm{r})
    \sim
    \frac{1}{|\bm{r}|^{(d-1)/2}}
    \mathrm{e}^{-a|\bm{r}|}
\end{align}
式\eqref{eq:ornstein_zernike}は{\bf Ornstein-Zernike(オルンシュタイン-ゼルニケ)の公式}と呼ばれ,二粒子の相関を表す関数(相関関数)の長距離における典型的な振る舞いを記述する.


\section*{アンケート}

この項目は成績とは無関係です.
\href{https://forms.gle/45Znp8u4y9uLkKJA8}{こちら(クリック)}のGoogleフォームよりアンケートにご協力ください.

%\bibliographystyle{unsrt}
%\bibliography{bib/ref}

\end{document}
