\documentclass[a4paper,12pt]{jsarticle}

%%%%%%%%%%%%%%%% パッケージの指定 %%%%%%%%%%%%%%%%
\usepackage{amsmath,amssymb}
\usepackage{bm}
\usepackage{braket}
\usepackage{cases}
\usepackage{dsfont}
\usepackage[dvipdfmx]{hyperref,graphicx}
\usepackage{pxjahyper}
\usepackage{fancyhdr}
\usepackage{here}
%%%%%%%%%%%%%%%%%%%%%%%%%%%%%%%%%%%%%%%%%%%%%%%%

%%%%%%%%%%% 式番号を(節番号.連番)に設定 %%%%%%%%%%%
\makeatletter
\@addtoreset{equation}{section}
\def\theequation{\thesection.\arabic{equation}}
\makeatother
%%%%%%%%%%%%%%%%%%%%%%%%%%%%%%%%%%%%%%%%%%%%%%%%

%%%%%%%%%%%%%%%% 文書の表題を設定 %%%%%%%%%%%%%%%%
\title{計算物理学II第1回レポート課題 解答例}
\author{名前\\学籍番号\\所属}
\date{\today}
%%%%%%%%%%%%%%%%%%%%%%%%%%%%%%%%%%%%%%%%%%%%%%%%

\begin{document}

\maketitle

\section{}
略.


\section{}

図\ref{fig:functions}は式\eqref{eq:problem_2}で与えられる関数$f(x)$を表している.
\begin{align}
\label{eq:problem_2}
    f(x)=
    \exp\left[\left(x\sin(\alpha x)\right)^{2}\right].
\end{align}

\begin{figure}
    \centering
    \includegraphics[scale=0.5]{fig/functions.pdf}
    \caption{課題2の関数のプロット}
    \label{fig:functions}
\end{figure}

\section{}

図\ref{fig:ibraki}は茨城県の人口の推移を表している.
図\ref{fig:ibraki}は国勢調査\cite{e-Stat}のデータに基づく.

\begin{figure}
    \centering
    \includegraphics[scale=1]{fig/ibaraki.pdf}
    \caption{課題3のプロット}
    \label{fig:ibraki}
\end{figure}

\section{}

\noindent {\bf (A)}
$q>0$の時,
\begin{align}
\label{eq:exp}
    \int^{\infty}_{0}\mathrm{d}t~\mathrm{e}^{-tq}
    =
    \left[-\frac{1}{q}\mathrm{e}^{-tq}\right]^{\infty}_{0}
    =
    \frac{1}{q}
\end{align}
となる.
\\

\noindent {\bf (B)}
被積分関数になっている指数関数の肩を平方完成すれば良い.
\begin{align}
\label{eq:int_k}
    \int^{\infty}_{-\infty}\mathrm{d}k_{i}
    ~\mathrm{e}^{-tk_{i}^{2}+\mathrm{i}k_{i}r_{i}}
    &=
    \int^{\infty}_{-\infty}\mathrm{d}k_{i}
    ~\exp\left[
        -t\left\{
            \left(
                k_{i}-\frac{\mathrm{i}r_{i}}{2t}
            \right)^{2}
            +
            \frac{r_{i}^{2}}{4t^{2}}
        \right\}
    \right]
    \nonumber\\
    &=
    \exp\left[
        -\frac{r_{i}^{2}}{4t}
    \right]
    \int^{\infty}_{-\infty}\mathrm{d}k_{i}
    ~\exp\left[
        -t\left(
            k_{i}-\frac{\mathrm{i}r_{i}}{2t}
        \right)^{2}
    \right]
    \nonumber\\
    &=
    \sqrt{\frac{\pi}{t}}
    \exp\left[
        -\frac{r_{i}^{2}}{4t}
    \right].
\end{align}
なお, 最後の等式はGauss積分による.
\\

\noindent {\bf (C)}
\begin{align}
\label{eq:correlator_g_exp}
    g(\bm{r})
    &=
    \left(\prod_{i=1}^{d}\int^{\infty}_{-\infty}\mathrm{d}k_{i}\right)
    \int^{\infty}_{0}\mathrm{d}t
    ~\mathrm{e}^{-t\left(|\bm{k}|^{2}+a^{2}\right)+\mathrm{i}\bm{k}\cdot\bm{r}}
    \nonumber\\
    &=
    \int^{\infty}_{0}\mathrm{d}t
    ~\mathrm{e}^{-a^{2}t}
    \prod_{i=1}^{d}
    \int^{\infty}_{-\infty}\mathrm{d}k_{i}
    ~\mathrm{e}^{-tk_{i}^{2}+\mathrm{i}k_{i}r_{i}}
    \nonumber\\
    &=
    \int^{\infty}_{0}\mathrm{d}t
    ~\mathrm{e}^{-a^{2}t}
    \prod_{i=1}^{d}
    \sqrt{\frac{\pi}{t}}
    \exp\left[
        -\frac{r_{i}^{2}}{4t}
    \right]
    \nonumber\\
    &=
    \pi^{d/2}
    \int^{\infty}_{0}\mathrm{d}t
    ~t^{-d/2}
    \exp\left[
        -a^{2}t-\frac{|\bm{r}|^{2}}{4t}
    \right].
\end{align}
ここで, $t=s/a^{2}$と変数変換すると,
\begin{align}
    g(\bm{r})
    &=
    \pi^{d/2}
    \int^{\infty}_{0}\frac{\mathrm{d}s}{a^{2}}
    ~\left(
        \frac{s}{a^{2}}
    \right)^{-d/2}
    \exp\left[
        -s-\frac{(a|\bm{r}|)^{2}}{s}
    \right]
    \nonumber\\
    &=
    \pi^{d/2}a^{d-2}
    \int^{\infty}_{0}\mathrm{d}s
    ~s^{-d/2}
    \exp\left[
        -s-\frac{(a|\bm{r}|)^{2}}{s}
    \right]
    \nonumber\\
    &=
    \pi^{d/2}a^{d-2}
    \frac{
        \frac{1}{2}\left(
            \frac{a|\bm{r}|}{2}
        \right)^{d/2-1}
    }{
        \frac{1}{2}\left(
            \frac{a|\bm{r}|}{2}
        \right)^{d/2-1}
    }
    \int^{\infty}_{0}\mathrm{d}s
    ~s^{-(d/2-1)-1}
    \exp\left[
        -s-\frac{(a|\bm{r}|)^{2}}{s}
    \right]
    \nonumber\\
    &=
    \pi^{d/2}a^{d-2}
    \frac{
        K_{d/2-1}(a|\bm{r}|)
    }{
        \frac{1}{2}\left(
            \frac{a|\bm{r}|}{2}
        \right)^{d/2-1}
    }
    \nonumber\\
    &=
    2\pi^{d/2}a^{d-2}
    \left(\frac{2}{a|\bm{r}|}\right)^{d/2-1}
    K_{d/2-1}(a|\bm{r}|).
\end{align}
\\

\noindent {\bf (D)}
$|\bm{r}|$が十分大きい場合, $K_{d/2-1}(a|\bm{r}|)$は, 
\begin{align}
    K_{d/2-1}(a|\bm{r}|)
    \sim
    \frac{\mathrm{e}^{-a|\bm{r}|}}{\sqrt{a|\bm{r}|}}
\end{align}
と振る舞うので,
\begin{align}
    g(\bm{r})
    &=
    2\pi^{d/2}a^{d-2}
    \left(\frac{2}{a|\bm{r}|}\right)^{d/2-1}
    K_{d/2-1}(a|\bm{r}|)
    \nonumber\\
    &\sim
    2\pi^{d/2}a^{d-2}
    \left(\frac{2}{a|\bm{r}|}\right)^{d/2-1}
    \frac{\mathrm{e}^{-a|\bm{r}|}}{\sqrt{a|\bm{r}|}}
    \nonumber\\
    &=
    2^{d/2}\pi^{d/2}a^{d-2}
    \frac{\mathrm{e}^{-a|\bm{r}|}}{\sqrt{a|\bm{r}|}^{d-1}}
\end{align}
となり, $g(\bm{r})$の$|\bm{r}|$依存性は,
\begin{align}
    g(\bm{r})
    \sim
    \frac{\mathrm{e}^{-a|\bm{r}|}}{|\bm{r}|^{(d-1)/2}}
\end{align}
で与えられる.


\bibliographystyle{unsrt}
\bibliography{bib/ref}

\end{document}
