\documentclass[a4paper,12pt]{jsarticle}

%%%%%%%%%%%%%%%% パッケージの指定 %%%%%%%%%%%%%%%%
\usepackage[top=20truemm,bottom=40truemm,left=40truemm,right=40truemm]{geometry}
\usepackage{amsmath,amssymb}
\usepackage{bm}
\usepackage{braket}
\usepackage{cases}
\usepackage{dsfont}
\usepackage[dvipdfmx]{hyperref,graphicx}
\usepackage{pxjahyper}
\usepackage{fancyhdr}
\usepackage{here}
\usepackage{listings}
%%%%%%%%%%%%%%%%%%%%%%%%%%%%%%%%%%%%%%%%%%%%%%%%

%%%%%%%%%%% 式番号を(節番号.連番)に設定 %%%%%%%%%%%
\makeatletter
\@addtoreset{equation}{section}
\def\theequation{\thesection.\arabic{equation}}
\makeatother
%%%%%%%%%%%%%%%%%%%%%%%%%%%%%%%%%%%%%%%%%%%%%%%%

%%%%%%%%%%%%%% サンプルコード表示設定 %%%%%%%%%%%%%%
\renewcommand{\lstlistingname}{ソースコード}
\lstset{
  basicstyle={\ttfamily},
  identifierstyle={\small},
  commentstyle={\smallitshape},
  keywordstyle={\small\bfseries},
  ndkeywordstyle={\small},
  stringstyle={\small\ttfamily},
  frame={tb},
  breaklines=true,
  columns=[l]{fullflexible},
  numbers=left,
  xrightmargin=0zw,
  xleftmargin=3zw,
  numberstyle={\scriptsize},
  stepnumber=1,
  numbersep=1zw,
  lineskip=-0.5ex
}
%%%%%%%%%%%%%%%%%%%%%%%%%%%%%%%%%%%%%%%%%%%%%%%%

%%%%%%%%%%%%%%%% 文書の表題を設定 %%%%%%%%%%%%%%%%
\title{計算物理学II 第2回レポート課題}
%\author{氏名\\学籍番号\\所属}
\date{提出期限 : 2025年11月28日}
%%%%%%%%%%%%%%%%%%%%%%%%%%%%%%%%%%%%%%%%%%%%%%%%

\begin{document}

\maketitle

以下の課題\ref{sec:problem_1},\ref{sec:problem_2},\ref{sec:problem_3},\ref{sec:problem_4}に取り組み,その結果を\LaTeX でレポートにまとめよ.
なお,以下の点に留意せよ.
\begin{itemize}
    \item レポートにはタイトルを付け,氏名,学籍番号,所属,レポート作成日を記載すること.
    \item レポート作成時に,この.pdfファイルのソースファイル(\texttt{lecture6/report2/main.tex})を活用しても構わない.
    \item 読みやすく,体裁の整ったレポート作成を心がけて欲しい.
    \item 作成したレポートの.pdfファイルと{\bf 課題\ref{sec:problem_1},課題\ref{sec:problem_2},課題\ref{sec:problem_4}で使用したソースコード(.pyファイル)}をmanabaに提出すること.
\end{itemize}

\section{}
\label{sec:problem_1}

Leibniz(ライプニッツ)の公式,
\begin{align}
    \sum_{n=0}^{\infty}\frac{(-1)^{n}}{2n+1}
    =
    \frac{\pi}{4}
\end{align}
を使って,$\pi$の近似値を次のように得ることを考える.
\begin{align}
    \pi
    \simeq
    4\sum_{n=0}^{N}\frac{(-1)^{n}}{2n+1}.
\end{align}
$N=10^{5}$の時の$\pi$の近似値を求めよ.


\section{}
\label{sec:problem_2}

Wallis(ウォリス)の公式
\begin{align}
    \prod_{n=1}^{\infty}
    \left(
        \frac{2n}{2n-1}
        \cdot
        \frac{2n}{2n+1}
    \right)
    =
    \frac{\pi}{2}
\end{align}
を使って,$\pi$の近似値を次のように得ることを考える.
\begin{align}
    \pi
    \simeq
    2\prod_{n=1}^{N}
    \left(
        \frac{2n}{2n-1}
        \cdot
        \frac{2n}{2n+1}
    \right).
\end{align}
$N=10^{5}$の時の$\pi$の近似値を求めよ.


\section{}
\label{sec:problem_3}

課題\ref{sec:problem_1},\ref{sec:problem_2}で求めた$N=10^{5}$での$\pi$の近似値について,どちらの方が真の$\pi$に近い結果を与えているか調べよ.
\footnote{\texttt{lecture\_material\_5.pdf}の67ページ参照.
}

\section{}
\label{sec:problem_4}

$f(x)$および$g(x)$として,異なる初等関数を自由に二つ選び,以下の課題に取り組んでみよ.\\

\begin{enumerate}
    \item[(1)] 自分の選んだ初等関数を明記した上で,$f(x)$,$g(x)$のTaylor展開を書け.
\end{enumerate}

\begin{enumerate}   
    \item[(2)] 以下のように,次数$N$で打ち切られた$f(x)$のTaylor展開を使って$f(a)$の近似値を求めることを考える.
    ただし,$a$は$f(x)$のTaylor展開の収束半径内にある数値とし,自由に選んでよい.
    \begin{itemize}
        \item 与えられた$a$と$N$に対して,次数$N$で打ち切られたTaylor展開から$f(a)$の近似値を求めるプログラムを作成すること.
        例えば,ソースコード\ref{example}は,$f(x)={\rm e}^{x}$の場合のサンプルプログラムである.
        20行目以下では,定義した関数\texttt{taylor$\_$exp}の引数を\texttt{for}文で回すことで,$N=1,2,\cdots,10$までの結果が得られるようになっている.
        \item Pythonで初等関数を使う場合,\texttt{math}モジュールにそれらが用意されている.
        \texttt{math}モジュールで使える数学関数については,\url{https://docs.python.org/ja/3/library/math.html}を参照するとよい.
        \item 「\LaTeX 入門」の5節,あるいはこの.pdfファイルのソースファイルを参照し,得られた結果を表\ref{tab:sample}のようにまとめよ.
        ただし,下の行に行くほど打ち切り次数$N$の値が大きくなるように並べること.
        なお,$N$の値は好きに選べばよいが,最低でも三つは選んで表に載せること.
        また,一番下の行には$N=\infty$の場合,すなわち$f(a)$の厳密な値を記載せよ.
    \end{itemize}
\end{enumerate}

\begin{enumerate}   
    \item[(3)] $g(x)$についても問(2)と同様のことを行え.
    \footnote{
    すなわち,次数$N$で打ち切られた$g(x)$のTaylor展開を使って$g(b)$の近似値を求め,厳密な値へ近づいていくことを確認せよ.ただし,$b$は$g(x)$のTaylor展開の収束半径内にある数値とし,自由に選んでよい.
    }
\end{enumerate}

\begin{table}[H]
    \centering
    \caption{$f(a)$のTaylor展開による近似}
    \begin{tabular}{c|c}
        \hline
        打ち切り次数$N$ & $f(a)$の近似値  \\
        \hline\hline
        2 & 得られた近似値 \\ 
        4 & 得られた近似値 \\ 
        6 & 得られた近似値 \\ 
        8 & 得られた近似値 \\ 
        10 & 得られた近似値 \\ 
        $\infty$ & 厳密な値 \\ 
        \hline
    \end{tabular}
    \label{tab:sample}
\end{table}



\clearpage

\begin{lstlisting}[caption=指数関数の場合の例,label=example]
import math

def taylor_exp(value_x,num_terms):
    acc = 0
    num = 1
    den = 1

    for index in range(num_terms):
        nextTerm = num / den
        acc = acc + nextTerm
        num = num * value_x
        den = den * (index+1)

    return acc

value_x = 0.1
exact = math.exp(value_x)
print("Exact = ",exact)

for num_terms in range(1,10):
    approximation = taylor_exp(value_x,num_terms)
    print(num_terms,approximation,math.fabs(exact-approximation))
\end{lstlisting}

%\bibliographystyle{unsrt}
%\bibliography{bib/ref}

\end{document}
