\documentclass[a4paper,12pt]{jsarticle}

%%%%%%%%%%%%%%%% パッケージの指定 %%%%%%%%%%%%%%%%
\usepackage[top=20truemm,bottom=40truemm,left=40truemm,right=40truemm]{geometry}
\usepackage{amsmath,amssymb}
\usepackage{bm}
\usepackage{braket}
\usepackage{cases}
\usepackage{dsfont}
\usepackage[dvipdfmx]{hyperref,graphicx}
\usepackage{pxjahyper}
\usepackage{fancyhdr}
\usepackage{here}
\usepackage{listings}
%%%%%%%%%%%%%%%%%%%%%%%%%%%%%%%%%%%%%%%%%%%%%%%%

%%%%%%%%%%% 式番号を(節番号.連番)に設定 %%%%%%%%%%%
\makeatletter
\@addtoreset{equation}{section}
\def\theequation{\thesection.\arabic{equation}}
\makeatother
%%%%%%%%%%%%%%%%%%%%%%%%%%%%%%%%%%%%%%%%%%%%%%%%

%%%%%%%%%%%%%% サンプルコード表示設定 %%%%%%%%%%%%%%
\renewcommand{\lstlistingname}{ソースコード}
\lstset{
  basicstyle={\ttfamily},
  identifierstyle={\small},
  commentstyle={\smallitshape},
  keywordstyle={\small\bfseries},
  ndkeywordstyle={\small},
  stringstyle={\small\ttfamily},
  frame={tb},
  breaklines=true,
  columns=[l]{fullflexible},
  numbers=left,
  xrightmargin=0zw,
  xleftmargin=3zw,
  numberstyle={\scriptsize},
  stepnumber=1,
  numbersep=1zw,
  lineskip=-0.5ex
}
%%%%%%%%%%%%%%%%%%%%%%%%%%%%%%%%%%%%%%%%%%%%%%%%

%%%%%%%%%%%%%%%% 文書の表題を設定 %%%%%%%%%%%%%%%%
\title{計算物理学II 第3回レポート課題}
%\author{氏名\\学籍番号\\所属}
\date{提出期限 : 2024年12月19日}
%%%%%%%%%%%%%%%%%%%%%%%%%%%%%%%%%%%%%%%%%%%%%%%%

\begin{document}

\maketitle

以下の課題\ref{sec:problem_fx},\ref{sec:problem_plot},\ref{sec:prablem_collatz},\ref{sec:problem_pearson}に取り組み,その結果を\LaTeX でレポートにまとめよ.
なお,以下の点に留意せよ.
\begin{itemize}
    \item レポートにはタイトルを付け,氏名,学籍番号,所属,レポート作成日を記載すること.
    \item レポート作成時に,この.pdfファイルのソースファイル(\texttt{lecture6/report2/main.tex})を活用しても構わない.
    \item 読みやすく,体裁の整ったレポート作成を心がけて欲しい.
    \item 作成したレポートの.pdfファイルと{\bf 課題\ref{sec:problem_fx},課題\ref{sec:problem_plot},課題\ref{sec:problem_pearson}で使用したソースコード(.pyファイル)}をmanabaに提出すること.
\end{itemize}

\section{}
\label{sec:problem_fx}

以下の関数を考える.ただし,$x\in\mathbb{R}$とする.
\begin{align}
\label{eq:naive}
    f(x)
    =
    \frac{1}{\sqrt{x^{2}+1}-x}.
\end{align}
この関数の値$f(x)$を計算するプログラムを以下の二通りの方法で書け.\\

\begin{enumerate}
    \item 式\eqref{eq:naive}の右辺である$1/(\sqrt{x^{2}+1}-x)$そのものを戻り値とする関数を書く.
    \item $f(x)$の右辺を,
    \begin{align}
        f(x)
        =
        \frac{\sqrt{x^{2}+1}+x}{(\sqrt{x^{2}+1}-x)(\sqrt{x^{2}+1}+x)}
        =
        \sqrt{x^{2}+1}+x,
    \end{align}
    と変形した上で,$\sqrt{x^{2}+1}+x$を戻り値とする関数を書く.\\
\end{enumerate}


この時,$x=10^{n}~(n=1,2,3,4,5,6,7)$に対して,上記の二通りの方法で$f(x)$を計算し,計算結果とその結果に対する簡単な考察を与えよ.
\footnote{余力のある人は,$n=8$でのプログラムの挙動も調べてみよ.とあるエラーが起こるはずである.なぜそのようなエラーが発生するのかを考えてみよ.}

\section{}
\label{sec:problem_plot}

以下の関数を考える.ただし,$x\in\mathbb{R}$とする.
\begin{align}
\label{eq:fx}
    f(x)=
    \frac{1-\cos x}{x^{2}}.
\end{align}\\

\noindent {\bf (A)}
$f(x)$のグラフを描け.\\

\noindent {\bf (Hint)} 

\begin{itemize}
    \item 第2回演習で配布したPythonスクリプト(\texttt{lecture2/src/sample\_plot\_functions.py})を参考にすると良い.ただし,$x$の値は,$x=0.1\times i~(i=1,2,\cdots,100)$などとせよ.
\end{itemize}

\noindent {\bf (B)}
$x=1.2\times 10^{-8}$における$f(x)$の値を数値計算せよ.
前問{\bf (A)}で描いたグラフを踏まえ,あるいは$\lim_{x\to0} f(x)$の極限値を考え,数値計算の結果が妥当かどうかを考察せよ.
妥当でない結果が得られている場合には,正しい結果を得るためにどのような改善策があり得るかを考え,その改善策が有効かどうかを検証せよ.

\section{}
\label{sec:prablem_collatz}

以下の漸化式で定義される数列$\{a_{n}\}_{n=1,2,\cdots}$を考える.
ただし,$a_{1}$は正の整数とする.
\begin{align}
    a_{n+1}
    =
    \begin{cases}
        a_{n}/2 &\quad\quad {\rm if}~n\in2\mathbb{Z},\\
        3a_{n}+1 &\quad\quad {\rm if}~n\notin2\mathbb{Z}.
    \end{cases}
\end{align}
Collatz予想によれば,任意の正の整数$a_{1}$に対して,$a_{c}=1$となるような整数$c$が存在するとされている.\\

\noindent {\bf (A)}
$a_{1}=100$の場合の整数値$c$を求めよ.
また,横軸を$n(=1,2,\cdots,c)$,縦軸を$a_{n}$として,数列$\{a_{n}\}$の値を図示せよ.\\

\noindent {\bf (B)}
前問{\bf (A)}とは異なる$a_{1}$を自分で設定し,横軸を$n(=1,2,\cdots,c)$,縦軸を$a_{n}$として,数列$\{a_{n}\}$の値を図示せよ.

\section{}
\label{sec:problem_pearson}

二つのデータセット$x=(x_{1},x_{2},\cdots,x_{n})$と$y=(y_{1},y_{2},\cdots,y_{n})$に対し,以下の式で定義される$r$をPearson(ピアソン)相関係数と呼ぶ.
\begin{align}
\label{eq:pearson}
    r=
    \frac{\sum_{i=1}^{n}(x_{i}-\bar{x})(y_{i}-\bar{y})}{(n-1)\sigma_{x}\sigma_{y}}
\end{align}
ここで,$\bar{x}$と$\bar{y}$は$x$と$y$の平均であり,$\sigma_{x}$と$\sigma_{y}$は標準偏差である.
$r$の値から,二つのデータセットの間の相関が分かる.
例えば,$r=1$の場合,二つのデータは同じ方向に変化(片方が増加すればもう片方も増加)する.このことを正の相関があるという.
反対に,$r=-1$の場合,二つのデータは逆の方向に変化(片方が増加すればもう片方は減少)する.このことを負の相関があるという.
また,$r=0$の場合,二つのデータの間には相関がないことを意味する.

\texttt{lecture8/report3/data}の中に\texttt{data1},\texttt{data2},\texttt{data3}という名前の三つの\texttt{.csv}ファイルが格納されている.
これらの\texttt{.csv}ファイルにはあるルールに従って生成された乱数のデータが保存されている.
そこで,各々の\texttt{.csv}ファイルに対して,一列目をデータセット$x$,二列目をデータセット$y$として式\eqref{eq:pearson}のPearson相関係数を計算せよ.
さらに,データセット$x$とデータセット$y$を散布図として可視化し,Pearson相関係数が正しく計算されていることを確認せよ.
レポートには,Pearson相関係数と散布図を掲載し,それらがどの\texttt{.csv}ファイルから得られたものか明記すること.\\

\noindent {\bf (Hint)} 

\begin{itemize}
    \item \texttt{lecture$\_$material$\_$8.pdf}のp.17と同様にすれば,\texttt{.csv}ファイルからデータを取得するコードが書ける.
    \item Pearson相関係数を計算するには平均と標準偏差の計算が必要である.これらの量を定義通り計算するコードを書いても良いが,\texttt{statistics}モジュールを\texttt{import}すれば,これらの量を即座に得ることができる.例えば,データセット$x$を\texttt{xList}というリストで管理した場合,\texttt{statistics.mean(xList)}で$\bar{x}$を,\texttt{statistics.stdev(xList)}で$\sigma_{x}$を得ることができる.
    \item 散布図の作成には,\texttt{matplotlib.pyplot}モジュールを使うと良い.データセット$x$と$y$をそれぞれ\texttt{xList}と\texttt{yList}というリストで管理した場合,ソースコード\ref{scatter}のようにすれば散布図を作成することができる.
    \item ソースコード\ref{scatter}の4行目で,\texttt{ax.set$\_$xlabel}の引数を\texttt{"X"}ではなく\texttt{r"\$X\$"}と書くことで\LaTeX の表記を使うことができる.
\end{itemize}

\begin{lstlisting}[caption=散布図を描くためのPythonスクリプト例,label=scatter]
fig = plt.figure()
ax = fig.add_subplot(1,1,1)
ax.scatter(xList, yList)
ax.set_xlabel("X")
ax.set_ylabel("Y")
fig.savefig("scatter.pdf")  
\end{lstlisting}


\section*{アンケート}

この項目は成績とは無関係です.
\href{https://forms.gle/e9HvrwikYkDX5mrv7}{こちら(クリック)}のGoogleフォームよりアンケートにご協力ください.

%\bibliographystyle{unsrt}
%\bibliography{bib/ref}

\end{document}
